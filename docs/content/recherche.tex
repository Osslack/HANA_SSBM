\chapter{Recherche}

Eine Datenbank hat die folgenden Performanz Faktoren\footnote{Vgl. \cite{sap:hana:performance_guide}[Seite 6]}:

\begin{itemize}
	\item Resourcen $\rightarrow$ CPU, Hauptspeicher, Festplatte
	\item Größe und Wachstum der Datenstrukturen
	\item Transaktionen
	\item Sicherheit, Autorisierung, und Lizensierung
	\item Konfiguration
\end{itemize}

\section{Permanent Langsames System}
Ein langsames System kann von vielen Faktoren abhängen.
Oft liegt es allerdings an zu wenigen Resourcen.
Dies kann mittels \verb+Administration > Overview+ oder \\
\verb+Administration > Performance > Load+ überprüft werden,
idem dort eine konstant hohe CPU, Hauptspeicher oder Netzwerk Auslastung angezeigt wird.

Dies kann auch mittels eines Betriebssystem Tools wie HTOP
analysiert werden.\footnote{Vgl. \cite{sap:hana:performance_guide}[Seite 8]}

% Wenn andere Queries laufen kann es sein dass HANA paging betreibt und ein Performanz test unnötig wäre.

%TODO Statement Performance Analysis

\section{Lizenzprobleme}
Es kann sein dass die Größe an speicher durch die Lizenz limitiert ist. (Siehe \cite{sap:hana:performance_guide}[Seite 9])

\section{Slow System-wide Performance}
Runtime Dump ausführen um zu analysieren.
\verb+\var\log\messages+ überprüfen um OS level Probleme zu erkennen.
\cite{sap:hana:performance_guide}[Seite 10]


\section{Hauptspeicher informationen}

\verb+Administration > Overview+
\verb+Configuration and Monitoring > Open Memory Overview+
Siehe \cite{sap:hana:performance_guide}[Seite 21]


%TODO Parallel Query execution
\cite{sap:hana:performance_guide}[Seite 40] $\rightarrow$ Parallele SQL Statement

%TODO IO

%TODO Delta Storage

%TODO Blocked Transaktionen

%TODO SQL Plan Cache Analysis
