\chapter{Performance Verbesserung durch Indizes}

Indizes können die Suche nach bestimmten Feldern beschleunigen, indem eine seperate Struktur angelegt wird, die schneller, in der Regel sogar logarithmisch, durchsucht werden kann.
%https://de.wikipedia.org/wiki/Datenbankindex
Die Auswirkung der grundlegenden Indizes, die bereits teil des SSBM-Schemas sind, wurde bereits in \ref{sec:basic_indizes} untersucht. 
Im folgenden sollen mögliche Indexkandidaten anhand der Query-Execution-Pläne identifiziert werden und deren Auswirkung auf die Laufzeit untersucht werden.

Alle Laufzeiten in diesem Kapitel wurden auf folgender Hardware getestet:
\begin{itemize}
    \item Host: Intel Xeon 1230 V3, 16 GB DDR3 1600MHz, 
    \item VM: 4 Kerne, 8 GB
\end{itemize}

Indexkandidaten:

D_YEAR bzw. alle Sachen aus DIM_DATE -> Q1er und Q4.2/3 2.500 auf 660 bis 7
S_Region -> q2.1/2, q (3.1),q4.1/2 In der Regel weniger als 25\% Treffermenge.
S_Nation -> q3.2, q4.3 -> 2000 auf 74 Einträge
C_Nation -> q3.2 30.000 auf 1.200

Mit Ausnahme von C_Nation handelt es sich nur um sehr kleine Tabellen.




