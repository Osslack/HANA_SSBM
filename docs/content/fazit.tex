\chapter{Fazit}
Generell lässt sich sagen, dass der Columnstore für die im SSBM Benchmark geschaffenen Bedingungen deutlich besser geeignet ist.
 In Zahlen bedeutet dies, dass der Columnstore im Schnitt mehr als 4 mal so schnell ist, wie der Rowstore.

Die exemplarische Analyse der Query Execution Pläne für Query 4.3 hat gezeigt, dass der große zeitliche Unterschied zwischen Row- und 
Columnstore darauf zurückzuführen ist, dass der Query im Columnstore paralellisiert abläuft.
Hier gibt es außerdem eine klare Trennung zwischen Filtern und Joinen, beim Rowstore ist dies enger miteinander verknüpft. 


Durch Indizes kann der Benchmark in beiden Fällen nochmals teils deutlich beschleunigt werden, allerdings gab es besonders beim Rowstore auch Fälle, 
wo bestimmte Queries durch die Indizes extrem verlangsamt wurden. Indizes sollten also bewusst eingesetzt und auf ungewünschte Nebeneffekte geprüft werden.

Außerdem hat sich gezeigt, dass die optimierte OLAP-Engine um Längen schneller ist, also die schnellste Variante der Column-Engine. In der Praxis
würde es sich also durchaus lohnen, seine Abfragen als Analytical Views zu realisieren, um diese Engine nutzen zu können. Für Rowstores ist die Egine leider nicht verfügbar.

Desweiteren hat sich gezeigt, dass der Columnstore immens von mehr CPU-Kernen profitiert, da hier sehr stark paralellisiert werden kann. 
Die Messwerte lassen auf einen quadratischen Zuwachs der Ausführungszeit abhängig zur Anzahl der CPU-Kerne schliessen. Durch geringe Mengen an 
Hauptspeicher nahm die Stabilität des Benchmarks stark ab, was vermutlich darauf zurückzuführen ist, dass bei reduziertem RAM externe Faktoren (Cache-Miss, Zugriffszeit) deutlicher zum tragen kommen. 

Beim Rowstore hat sich gezeigt, dass hier sowohl CPU als auch RAM von größerer Bedeutung für die Laufzeit sind. 
Während im Columnstore automatisch Optimierungen, wie Indexierung und Kompression stattfinden, können diese im Rowstore womöglich nur bei ausreichend RAM durchgeführt werden. 

Zusammenfassend lässt sich sagen, dass die oft im Netz zu findende Empfehlung, hauptsächlich auf Columnstores zu setzen, durch unsere Ergebnisse unterstützt wird.
