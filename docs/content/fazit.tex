\chapter{Fazit}\label{chapter:fazit}
Generell lässt sich sagen, dass der Columnstore für die im SSBM Benchmark
geschaffenen Bedingungen deutlich besser geeignet ist.
In Zahlen bedeutet dies, dass der Columnstore im Schnitt mehr als 4 mal so schnell ist,
wie der Rowstore.

Die exemplarische Analyse der Query Execution Pläne für Query 4.3 hat gezeigt,
dass der große zeitliche Unterschied zwischen Row- und 
Columnstore darauf zurückzuführen ist, dass die Queries im Columnstore paralellisiert ablaufen.
Hier gibt es außerdem eine klare Trennung zwischen Filter und Join,
beim Rowstore ist dies enger miteinander verknüpft. 


Durch Indizes kann der Benchmark in beiden Fällen nochmals teils deutlich beschleunigt werden,
allerdings gibt es besonders beim Rowstore auch Fälle,
wo bestimmte Queries durch die Indizes verlangsamt wurden.
Indizes sollten also bewusst eingesetzt und auf ungewünschte Nebeneffekte geprüft werden.

Außerdem hat sich gezeigt, dass die optimierte OLAP-Engine deutlich schneller ist,
also die schnellste Variante der Column-Engine.
In der Praxis kann es von Vorteil sein, Abfragen als Analytical Views zu realisieren,
um diese Engine nutzen zu können. Für Rowstores ist die Engine leider nicht verfügbar.

Des Weiteren hat sich gezeigt, dass der Columnstore von mehr CPU-Kernen profitiert,
da hier sehr stark parallelisiert werden kann. 
Die Messwerte lassen auf einen quadratischen Zuwachs der Ausführungszeit
abhängig zur Anzahl der CPU-Kerne schliessen.
Durch reduzieren des Hauptspeicher nahm die Stabilität des Benchmarks stark ab,
was vermutlich darauf zurückzuführen ist, dass bei reduziertem RAM
externe Faktoren (Cache-Miss, Zugriffszeit) zum tragen kommen. 

Beim Rowstore hat sich gezeigt, dass hier sowohl CPU als auch RAM von größerer
Bedeutung für die Laufzeit sind. 
Während im Columnstore automatisch Optimierungen, wie Indexierung und Kompression stattfinden,
können diese im Rowstore nur bei ausreichend RAM durchgeführt werden. 
Außerdem wird im Rowstore die Performance durch wenige Kerne aufgrund der fehlenden Parallelität umso mehr eingeschränkt, da Lesezugriffe auf die Ausführung der CPU warten müssen und umgekehrt. 

Zusammenfassend lässt sich sagen, dass die oft im Netz zu findende Empfehlung,
hauptsächlich auf Columnstores zu setzen, durch unsere Ergebnisse unterstützt wird.
