\chapter{Auswertung Benchmark}
Im folgenden werden die Ergebnisse des Benchmarks ausgewertet.
Dazu wird zu erst die Gesamtlaufzeit des Benchmarks in Abschnitt \ref{auswertung:generell} analysiert. Dabei wird zwischen zeilenbasierten und spatenbasierten
Tabellen unterschieden.
Anschließend wird in Abschnitt \ref{auswertung:queries} auf die Laufzeit der
einzelnen Unterabfragen des Benchmarks eingegangen.
Dabei soll untersucht werden welche Abfragen besonders schnell sind.
In Abschnitt \ref{auswertung:basic_indizes} und \ref{auswertung:hardware}
wird der Einfluss von Indizes bzw. unterschiedlicher Hardwarekonfigurationen
untersucht.

\section{Gesamtlaufzeit des Benchmarks}\label{auswertung:generell}

%TODO ROW, COLUMN
% Grundlegende Statistische messwerte
% Performanceentwicklung des benchmarks
% Mehr dazu in \ref{auswertung:queries}

\subsection{Vergleich Zeilenbasiert vs. Spaltenbasiert}\label{auswertung:row_vs_col}
%TODO
% Hypothese Spaltenbasiert ist schneller
% Welches is schneller
% Wie unterscheidet sich die Standardabweichung
% Bestätigung der Hypothese

\section{Vergleich der SSBM Queries}\label{auswertung:queries}

%TODO
% Vergleich von q1, q2, q3, q4 etc.
% Diagramm
% Welche ist die schnellste und warum?
% Stabilität einzelner Queries

\newpage
\section{Einfluss der grundlegenden Indizes}\label{auswertung:basic_indizes}
Die Benchmarkergebnisse in diesem Abschnitt wurden auf folgender Hardware erzielt:
\begin{itemize}
    \item Host: i7 7700k, 16GB DDR4 RAM
    \item VM: 6 Kerne @90\%, 8GB RAM
\end{itemize}
%TODO
% Row with or without indices
% Column with or without indices

\subsection{Grundlegende Untersuchung für Column-Store}

\begin{table}[H]
\centering
    \begin{tabularx}{10cm}{lrrr}
        \toprule
        Merkmal             &   Col[ms]    &    Col Index[ms] & Abweichung\footnote{Abweichung von \enquote{mit Index} in Relation zu \enquote{ohne Index}.}[\%]\\
        \toprule
        Samples             &   250        &   250      &        \\
        \midrule    
        Max                 &   859        &   697      & -18.8\%\\
        Min                 &   726        &   546      & -24.7\%\\
        Median              &   752        &   574      & -23.6\%\\
        Average             &   755        &   578      & -23.4\%\\
        Standard Deviation  &   18         &   19       & +5.5\%\\
        Total               &   188672     &   144605   & -23.3\%\\
        \bottomrule
    \end{tabularx}
\caption{Vergleich der Ergebnisse mit und ohne grundlegende Indizes für Column-Store.}
\label{tab:basic_index_col}
\end{table}

Durch hinzufügen der grundlegenden Indizes wurde der Benchmark für Colum-Stores sowohl im Schnitt als auch im Median deutlich schneller.
Im Schnitt wurde er 23.4\%, im Median um 23,6\% schneller. Die Standardabweichung hat sich jedoch nur geringfügig verändert, die Werte streuen relativ gesehen also gleich stark wie zuvor. Folglich konnten Mindest- und Maximallaufzeit auch deutlich reduziert werden. 


Diese Ergebnisse sind in sofern interessant, als dass Column-Stores eigentlich bereits einen quasi-Index für die Spalten haben und deshalb nicht so deutlich von Indizes profitieren sollten, siehe \ref{sec:col_store}.
Um herauszufinden, warum trotzdem so eine deutliche Verbesserung merkbar ist, wird der Query-Execution Plan des Subqueries mit der deutlichsten Verbesserung untersucht.

\subsubsection{Verbesserung der Laufzeit für einzelne Query(gruppen)}

\begin{table}[!ht]
    \centering
    \begin{tabularx}{14cm}{lrrr}
        \toprule
        Benchmarkgruppe & Col[ms]   & Col Index[ms] & Laufzeitreduzierung[ms|\%]\\
        \toprule
        Q1              & 104.7       & 68.5            & 36.2 | 34.5\%\\
        Q2              & 62.1        & 59.7            & 2.4  | 3.8\%\\
        Q3              & 96.2        & 54.8            & 41.4 | 40.8\%\\
        Q4              & 112.4       & 106.3           & 6.1  | 5.4\%\\
        \bottomrule
    \end{tabularx}
\caption{Durchschnittslaufzeit für jede Benchmarkgruppe für Column-Store.}
\end{table}

Die deutlichste Verbesserung ist bei den Queries in der Gruppe 3 festzustellen. Hier hat sich die Laufzeit um 40\% reduziert. Die Queries dieser Gruppe werden im Detail untersucht, um einen geeigneten Kandidaten für die Analyse des Execution-Plans zu finden.
Hier ist besonders bei Query 3.3 und 3.4 eine extreme Verbesserung ersichtlich. Die Laufzeit wurde hier um knapp mehr als 90\% reduziert. Woher diese immense Verbesserung kommt, soll im Folgenden durch die Analyse des Execution-Plans von Query 3.4 geklärt werden.


\begin{table}[!ht]
    \centering
    \begin{tabularx}{13cm}{lrrr}
        \toprule
        Benchmark           & Col[ms]       & Col Index[ms] & Laufzeitreduzierung[ms|\%]   \\
        \toprule
        Q3.1                & 31.3          & 31.6          & -0.3 | -0.9\%                \\
        Q3.2                & 24.0          & 18.9          & 5.1 | 21.2\%                 \\
        Q3.3                & 21.3          & 2.1           & 19.2 | 90.1\%                \\
        Q3.4                & 20.5          & 1.6           & 18.9 | 92.1\%                \\
        \bottomrule
    \end{tabularx}
\caption{Durchschnittslaufzeit für Benchmarkgruppe 3 für Column-Store.}
\end{table}

\subsubsection{Analyse des Query-Execution-Plans für Query 3.4}

Bei der Analyse des Query-Execution Plans zeigt sich schnell, woher die große Verbesserung kommt. In Query 3.4 wird ein JOIN von Lineorder auf Customer, Supplier und Dim\_Date gemacht. 
Dieser JOIN erfolgt jeweils über den Foreign Key in Lineorder. Ohne Indizes ist dieser JOIN hauptverantwortlich für die Laufzeit des Querys.
Durch die grundlegenden Indizes wird auf jeden dieser Foreign Keys ein Index angelegt, weshalb der JOIN deutlich beschleunigt wird. Am meisten Einfluss hat hier der Index auf LO\_Suppkey.

Beide Execution-Pläne sind im Anhang zu finden, mit Index \hyperlink{q34I.1}{hier}, ohne Index \hyperlink{q34noI.1}{hier}. 

Query 3.1 nutzt zwar auch diese Foreign Keys, um die Tabellen zu JOINEN, allerdings sind hier viele Felder in der WHERE-Klausel und im GROUP BY, die nicht indiziert sind, nämlich S\_Region, D\_Year, C\_Nation und S\_Nation.

Auch bei Column-Stores scheinen sinnvoll angelegte Indizes also einen deutlichen Unterschied zu machen. 


\subsection{Grundlegende Untersuchung für Row-Store}

\begin{table}[!ht]
    \begin{tabularx}{10cm}{l|rr}
        \toprule
        Wert                & Row[ms] & Row Index[ms]\\
        \toprule
        Samples             & 250      & 250\\
        Min                 & 3353     & 2839\\
        Median              & 3519     & 3006\\
        Average             & 3539     & 3048\\
        Max                 & 4283     & 3394\\
        Standard Deviation  & 110      & 124\\
        Total               & 884960   & 762128\\ 
    \end{tabularx}
\caption{Vergleich der Ergebnisse mit und ohne grundlegende Indizes für Row-Store.}
\label{tab:basic_index_row}
\end{table}



\section{Auswirkung unterschiedlicher Hardwarekonfiguration}\label{auswertung:hardware}

%TODO
% Beschreibe setup bsp 4GB ram, 2CPU etc.
% Gibt es einen effekt?
% Wenn ja isolieren und begründen

\section{Vergleich zu anderen Datenbanksystemen}\label{auswertung:vergleich}

%TODO
%https://github.com/Osslack/HANA_SSBM/issues/27
%siehe auch ressourcen

