\chapter{Auswertung Benchmark}
Im folgenden werden die Ergebnisse des Benchmarks ausgewertet.
Dazu wird zu erst die Gesamtlaufzeit des Benchmarks in Abschnitt \ref{auswertung:generell} analysiert. Dabei wird zwischen zeilenbasierten und spatenbasierten
Tabellen unterschieden.
Anschließend wird in Abschnitt \ref{auswertung:queries} auf die Laufzeit der
einzelnen Unterabfragen des Benchmarks eingegangen.
Dabei soll untersucht werden welche Abfragen besonders schnell sind.
In Abschnitt \ref{auswertung:basic_indizes} und \ref{auswertung:hardware}
wird der Einfluss von Indizes bzw. unterschiedlicher Hardwarekonfigurationen
untersucht.

\section{Gesamtlaufzeit des Benchmarks}\label{auswertung:generell}

%TODO ROW, COLUMN
% Grundlegende Statistische messwerte
% Performanceentwicklung des benchmarks
% Mehr dazu in \ref{auswertung:queries}

\subsection{Vergleich Zeilenbasiert vs. Spaltenbasiert}\label{auswertung:row_vs_col}
%TODO
% Hypothese Spaltenbasiert ist schneller
% Welches is schneller
% Wie unterscheidet sich die Standardabweichung
% Bestätigung der Hypothese

\section{Vergleich der SSBM Queries}\label{auswertung:queries}

%TODO
% Vergleich von q1, q2, q3, q4 etc.
% Diagramm
% Welche ist die schnellste und warum?
% Stabilität einzelner Queries


\section{Einfluss der grundlegenden Indizes}\label{auswertung:basic_indizes}

%TODO
% Row with or without indices
% Column with or without indices

\section{Auswirkung unterschiedlicher Hardwarekonfiguration}\label{auswertung:hardware}

%TODO
% Beschreibe setup bsp 4GB ram, 2CPU etc.
% Gibt es einen effekt?
% Wenn ja isolieren und begründen

\section{Vergleich zu anderen Datenbanksystemen}\label{auswertung:vergleich}

%TODO
%https://github.com/Osslack/HANA_SSBM/issues/27
%siehe auch ressourcen

