\chapter{Setup}
Diese Kapitel beschreibt das grundlegende Setup, um mit der SAP HANA Datenbank, in Form einer virtuellen Maschine, arbeiten zu können.

\section{Virtuelle Machine}
\textbf{ToDo - Setup von Arwed beachten}


\section{\enquote{Eclipse}}
Damit verschiedene Operationen auf der SAP HANA Datenbank ausgeführt werden können, ist es ratsam die Entwicklungsumgebung \enquote{Eclipse} zum Ausführen von SQL Statements zu nutzen. \\Unter dem folgenden Link (\url{https://www.sap.com/developer/how-tos/2016/09/hxe-howto-eclipse.html}) ist beschrieben, welche Erweiterungen und Einstellungen in \enquote{Eclipse} vorgenommen werden müssen, um eine Verbindung zur Datenbank herzustellen.

\subsection{Query Execution Plan}
Um nachvollziehen zu können, in welcher Abfolge die SQL Statements von der HANA Datenbank verarbeitet werden, lassen sich Query Execution Pläne erstellen.
Dafür sind folgende Schritte notwendig:
\begin{enumerate}
	\item SQL Console öffnen \& Statement eingeben
	\item \enquote{Rechtsklick} im Context Fenster der SQL Console
	\item Wähle den Menüpunkt \enquote{Visualize Plan} $\rightarrow$ \enquote{Execute} aus.
	\item Der Query Execution Plan wird nun angezeigt.
\end{enumerate} 

\section{Daten Import}
Dieses Kapitel beschreibt, wie die SSBM-Benchmark-Daten in SAP HANA Datenbank geladen werden können. Dies kann entweder über die Entwicklungsumgebung Eclipse geschehen, oder über die Kommandozeile, indem \enquote{HDBSQL} verwendet wird.

\subsection{Eclipse}

\subsection{HDBSQL}
