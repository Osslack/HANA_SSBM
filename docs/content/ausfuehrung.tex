\chapter{Ausführung des Benchmarks}\label{chapter:run}

Das HANA Studio Plugin für Eclipse ermöglicht die direkte Ausführung von SQL-Code über die SQL-Console und würde damit ausreichen um die Schemata anzulegen, die Daten zu importieren und den Benchmark auszuführen. Werden nun jedoch Ansprüche wie das mehrfache Ausführen des Benchmarks unter unterschiedlichen Bedingungen gestellt, so ist offensichtlich, dass das händische Ausführen der einzelnen Schritte nachteilig ist. Eine elegantere Lösung ist die des Einsatzes von Skripten, welche Logik implementieren zur automatisierten Ausführung der Benchmarks. Dem Prozess der Ausführung des Benchmarks liegen dabei mehrere Gedanken zugrunde, welche in diesem Kapitel vorgestellt werden. 

\section{Ziele}

Die erwähnten Ansprüche an den kompletten Benchmark beziehen sich unter anderem auf seine \textbf{erleichtete Ausführung}. Wie im Kapitel zum Setup vorgestellt wurde, existiert ein Bash Skript {\glqq}run{\textunderscore}benchmark.bash{\grqq}, welches das zentrale Skript darstellt und dessen alleiniger Aufruf zur Ausführung des kompletten Benchmarks ausreicht. Somit ist die Komplexität der Ausführung für den Anwender reduziert. Nicht nur wird dadurch die Ausführung an sich erleichtert, auch die Installation des HANA Studio Plugins für Eclipse wird überflüssig, da jegliche SQL-Befehle automatisch über das Skript aus SQL-Dateien ausgeführt wird. Es sind also keine SQL-Kenntnisse für den Anwender notwendig, jeglich der Umgang mit Bash-Skripten. 

Dank der erleichterten Ausführung und Einrichtung der Benchmark-Umgebung ist es einfach den Benchmark auf \textbf{unterschiedlichen Test-Systemen} ausführen zu können. Durch die Variation der Test-Systeme können Faktoren wie die Anzahl zur Verfügung gestellter CPU Kerne oder RAM in ihrem Einfluss auf den Benchmark untersucht werden. Auf diese Aspekte wird im Folgekapitel näher eingegangen. 

Da die Evaluierung der Ergebnisse erst durchgeführt werden kann sobald alle Ergebnisse vorhanden sind (also nach Ausführung aller Benchmarks) muss eine Möglichkeit geschaffen werden, die Ergebnisse zwischenzuspeichern. Zu diesem Zwecke werden während der Durchführung der Benchmarks \textbf{Logs} angelegt, welche die Daten halten. 

Ein weiterer Aspekt ist die \textbf{Anzahl an Iterationen} innherhalb des kompletten Benchmarks. Viele Iterationen stellen den Ausschluss von Anomalien sicher und geben dem \textbf{Analyser} in der späteren Evaluierung zuverlässigere Werte. Dieser wird die Ergebnisse vergleichen und in einem einzigen Dokument erfassen. 

Nicht nur werden die Bedingungen für die Benchmarks variiert, sondern auch deren Inhalt. So werden unterschiedliche Konstellationen im Einsatz von \textbf{Row- und Columnstore} sowie \textbf{Indizes} durchgespielt. Genau wie die Variation des Test-Systems lassen sich dadurch essentielle Daten für den Analyser generieren. 

\section{Realisierung der Ziele}

Die eingesetzten Test-Systeme variieren folgendermassen: 
\begin{itemize}
	\item \textbf{RAM}: Es werden 6, 8 und 12 Gigabyte von 1.6 Tausend MHz DDR3 bis hin zu 3 Tausend MHz DDR4 RAM zur Verfügung gestellt. 
	\item \textbf{CPU}: Es werden 2, 4 und 6 virtuelle Kerne von 3.30GHz bis 4.2GHz zur Verfügung gestellt. 
\end{itemize}

Die Anzahl der Iterationen wird auf den Wert 250 festgelegt.

\newpage

\section{Durchführung}

\begin{wrapfigure}{r}{0.25\textwidth} 
    \centering{
    \includegraphics[scale=0.7]{images/Durchfuehrung3.png}
    \caption{Durchführung}\label{durchfuehrung}}
\end{wrapfigure}

Wie in \autoref{durchfuehrung} zu erkennen ist, lässt sich die Durchführung des Benchmarks unterteilen in die Schritte Schema Erzeugung, Daten Import, Index Erzeugung, Ausführung des Benchmarks, Speicherung der Daten ins Log und die Analyse durch den Analyser. Zur Übersichtlichkeit wird ein simplifizierter Prozess dargestellt, denn die eigentliche Durchführung involviert mehrere Unterschritte, die die angesprochende inhaltliche Variation des Benchmarks realisieren. 

Nach erfolgreicher Anmeldung über die Zugangsdaten zur HANA Instanz erfolgt zuerst der Benchmark auf Basis eines Columnstores ohne Indizies. Dazu werden zuerst die Daten importiert und das Schema für den Columnstore angelegt. Anschliessend werden über den Aufruf des Skriptes {\glqq}all{\textunderscore}benchmarks.bash{\grqq} die einzelnen Queries ausgeführt. 

Nach der Ausführung des ersten Benchmarks werden in drei Schritten Indizes hinzugefügt und jeweils erneut Benchmarks durchgeführt. Darauf schliesst sich der Wechsel zum Rowstore an, was zuerst das Anlegen des Schemas für den Rowstore und ein erneutes Importieren der Daten involviert. Die Schritte zum Ausführen des Benchmarks bei unterschiedlichen Indizes werden nun wiederholt. 

Eine Iteration des Skriptes resultiert damit in acht einzelnen Benchmarks. Die Daten aus den Logs werden im folgenden Schritt vom Analyser analysiert. Die Auswertung des Benchmarks bezieht sich vorallem auf folgende System-Konfiguration: 6 Kerne @ 4.2GHz bei 8 Gigybyte RAM. 

\section{Verwendung des Benchmarks}

Um möglichst gute Ergebnisse für den Benchmark zu erhalten ist es ein
Ziel den Benchmark mehrmals auszuführen. Um dies zu realisieren war es Teil dieser
Arbeit ein Script zu entwickeln welches den Benchmark automatisiert.
Da die HANA Express Datenbank in einer Suse Linux basierten virtuellen Maschine
betrieben werden kann, war es naheliegend dieses Script in BASH\cite{bash}
zu schreiben, um keine extra Compiler oder Interpreter in der virtuellen Maschine
installieren zu müssen.
Das resultierende Script kann unter \faGithub \url{https://github.com/Osslack/HANA_SSBM}
gefunden werden.

Durch ausführen des \href{https://github.com/Osslack/HANA_SSBM/blob/master/src/run_benchmark.bash}{run\_benchmark.sh} scripts kann der Benchmark gestarted werden.

Dabei geht das Script davon aus dass die HANA Datenbank mit den Standardeinstellungen
auf dem selben System installiert ist.
Standardmäßig versucht das Script mittels des SYSTEM Users auf das System zuzugreifen.
Möchte man nicht die Standardeinstellungen verwenden, muss das Script mittels \verb+run_benchmark.sh -v+ ausgeführt werden.
Dadurch bekommt man bei jedem Schritt die Möglichkeit, die Einstellung anzupassen.

Um den Benchmark anzupassen können neue Tests dem Script hinzugefügt werden.
Die Idee dabei ist, dass der Tester dem Framework sagt, wie seine
Benchmarks strukturiert sind und anschließend mittels SQL\cite{sql} und BASH
spezifische Queries ausführen und wiederholen kann.
Anschließend werden die Laufzeitdaten der SQL Queries als JSON\cite{json} in einer
Log Datei gespeichert.
Dann kann ein Analyst mit seiner favorisierten Programmiersprache
die Daten auswerten. In unserem Fall erfolgte die Auswertung in Python unter zu Hilfe nahme von Jupyter Notebook zur Visualisierung der Ergebnisse.

Dabei kann mittels des \verb+hdb_run_file_lite+ ein SQL-Script
zum Setup und Cleanup der Datenbank vor bzw. nach jedem Benchmark ausgeführt werden.

Mittels \verb+hdb_start_benchmark+ kann ein neuer Benchmark gestarted werden.
Dabei geht es nicht darum einen speziellen Benchmark zu starten.
Es sagt lediglich, dass ein neuer Benchmark beginnt. Dadurch kann das Script
das Log strukturieren.
Anschließend können mehrere SQL-Kommandos mittels \verb+all_benchmarks <path> <repetitions>+ ausgeführt werden.
Dabei gibt \verb+path+ an in welchem Ordner die SQL Dateien gespeichert sind.
Das Skript ließt alle SQL Dateien ein und führt sie \enquote{repetitions}-mal aus.
Zum Beenden des Benchmarks muss \verb+hdb_end_benchmark+ aufgerufen werden.
Alle Benchmarks müssen in einem \verb+hdb_init_log+ - \verb+hdb_finish_log+ Statement 
stehen damit sie in eine Log Datei geschrieben werden.
Ein Beispiel ist in \autoref{ausfuehrung:beispiel-benchmark} zu sehen.

\begin{lstlisting}[
	language=bash,
	caption={Beispiel Benchmark},
	label={ausfuehrung:beispiel-benchmark}
]
source ./hdbsql.bash
source ./all_benchmarks.bash

hdb_init_log

hdb_start_benchmark <benchmark name>
run_all_benchmarks <path> <repetitions>
hdb_end_benchmark

hdb_finish_log
\end{lstlisting}

Ein Beispiel einer resultierenden Log Datei ist in \autoref{exec:log-file} zu sehen.
Es wird jeder Benchmark als Attribut gespeichert welches aus einem Array Besteht.
Der Array besteht wiederrum aus einzelnen Arrays welche die Wiederholung des Benchmarks repräsentieren.
Ein einzelner Testlauf besteht aus mehreren Objekten. Jedes Objekt repräsentiert dabei
einen ausgeführten Befehl. Dieser hat einen Typ welcher z.B. \verb+exec_file+ oder 
\verb+inline_command+ sein kann.
Dieser Typ gibt an wie das Objekt zu lesen ist. So sagt zum Beispiel der Typ \verb+exec_file+
dass der Befehl durch das Ausführen einer SQL Datei gestartet wurde, wohingegen
der Typ \verb+inline_command+ sagt dass der SQL Befehl direkt als String
an die HANA Datenbank gesendet wurde. Dementsprechend kann ein Befehl vom Typ \verb+exec_file+
im Gegensatz zu einem Befehl des Typ \verb+inline_command+ einen Filenamen als Attribut haben.

Das Attribut \verb+times+ gibt in jedem Fall die Ausführungszeit des Befehls an.
Dabei kann sich eine Query in Teilqueries unterteilen, weshalb die einzelnen Laufzeiten
per Semikolon separiert in einem String gespeichert werden.
Ein Analyse Werkzeug müsste die einzelnen Laufzeiten parsen und summieren um die
Gesamtlaufzeit des einzelnen Befehls zu erhalten.
Die Zeiten sind in Mikrosekunden angegeben.

\colorlet{punct}{red!60!black}
\definecolor{background}{HTML}{EEEEEE}
\definecolor{delim}{RGB}{20,105,176}
\colorlet{numb}{magenta!60!black}

\lstdefinelanguage{json}{
    basicstyle=\normalfont\ttfamily,
    numbers=left,
    numberstyle=\scriptsize,
    stepnumber=1,
    numbersep=8pt,
    showstringspaces=false,
    breaklines=true,
    frame=lines,
    literate=
     *{0}{{{\color{numb}0}}}{1}
      {1}{{{\color{numb}1}}}{1}
      {2}{{{\color{numb}2}}}{1}
      {3}{{{\color{numb}3}}}{1}
      {4}{{{\color{numb}4}}}{1}
      {5}{{{\color{numb}5}}}{1}
      {6}{{{\color{numb}6}}}{1}
      {7}{{{\color{numb}7}}}{1}
      {8}{{{\color{numb}8}}}{1}
      {9}{{{\color{numb}9}}}{1}
      {:}{{{\color{punct}{:}}}}{1}
      {,}{{{\color{punct}{,}}}}{1}
      {\{}{{{\color{delim}{\{}}}}{1}
      {\}}{{{\color{delim}{\}}}}}{1}
      {[}{{{\color{delim}{[}}}}{1}
      {]}{{{\color{delim}{]}}}}{1},
}

\begin{lstlisting}[language=json, caption={Aufbau der Log Datei}, label={exec:log-file}]
{
	"column_benchmark_no_index": [
		[
			{
				"Type": "exec_file",
				"Filename": "./sql/benchmark/q1_bench/q1.sql",
				"times": " 88645;70697;39871;"
			},
			{
				"Type": "exec_file",
				"Filename": "./sql/benchmark/q1_bench/q1.1.sql",
				"times": " 44470;"
			}
		]
	]
}
\end{lstlisting}


