\chapter{Einleitung}
Im Rahmen der Vorlesung \enquote{Data Warehouse} soll eine Analyse des Star Schema Benchmarks mit der SAP HANA Datenbank durchgeführt werden. 
\\Heutzutage werden Entscheidungen von Unternehmen oftmals anhand von Kennzahlen getroffen. Da es allerdings noch häufig vorkommt, dass Datenbanken für analytische Zwecke (OLAP) und für Transaktionen (OLAP) getrennt voneinander behandelt werden, ist meist die Systemlandschaft komplex. Um in solchen Systemen Analysen ausführen zu können,  wird in ETL-Prozessen festgelegt, welche Daten für die Analyse in ein OLAP-System importiert werden. Der Nachteil solcher Anwendungen ist, dass der Datenbestand, der der Analyse zugrunde liegt, niemals vollständig ist. 
\\Durch die In-Memory Datenspeicherung der SAP HANA Datenbank liefert diese in sehr kurzer Zeit die geforderten Ergebnisse zurück. Es ist möglich, Daten innerhalb eines Row- bwz. Column-Stores in der Datenbank zu halten. Außerdem erzielt die SAP HANA Datenbank durch Anwendung verschiedener Algorithmen eine gute Datenkompression. 
\\Die SAP HANA ermöglicht es die Komplexität der Systemlandschaften zu reduzieren, indem OLAP und OLTP Funktionalität in einer Datenbank integriert werden. Dabei werden Echtzeitanalysen erstellt und als Entscheidungsgrundlage von Unternehmen genutzt.
\\Im folgenden Dokument sollen die Ausführungs- bzw. Antwortzeiten der SQL-Abfragen des Star Schema Benchmarks in der In-Memory Datenbank zwischen dem Row- bzw. Column-Store verglichen und analysiert werden.
