%!TEX root = ../dokumentation.tex

\pagestyle{empty}

% Dieser deutsche Teil wird nur angezeigt, wenn die Sprache auf Deutsch eingestellt ist.
\renewcommand{\abstractname}{\langabstract} % Text für Überschrift

% \begin{otherlanguage}{english} % auskommentieren, wenn Abstract auf Deutsch sein soll
\begin{abstract}\label{abstract}
	Die nachfolgende Arbeit betrachtet die Ausführung des Star Schema Benchmarks mit einer SAP HANA Datenbank. 
	Grundlegend werden einige Unterschiede von verschiedenen Speicherarten, Column- bzw. Rowstore, in In-Memory Datenbanktabellen aufgezeigt. 
	Insbesondere  werden Laufzeitunterschiede, Verbesserungsmöglichkeiten der Laufzeit mit Indizes, sowie die Auswirkungen von verschiedenen Hardwarevoraussetzungen 
	eines Coulmnstores und Rowstores betrachtet und analysiert. Hierbei hat sich gezeigt, dass der Columnstore in allen Belangen deutlich schneller ist, als der Rowstore.
	Durch Indizes lässt sich dieser sogar noch beschleunigen, wobei die schnellsten Durchläufe durch Nutzung der OLAP-Engine erzielt werden konnten.
\end{abstract}

